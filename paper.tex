\documentclass[12pt]{article} 

%\usepackage{ucs}
%\usepackage[utf8x]{inputenc}

%\usepackage[T1]{fontenc}
%\usepackage{textcomp}
%\renewcommand{\rmdefault}{ptm}
%\renewcommand{\sfdefault}{phv}

%\usepackage{charter}
%\usepackage{mathdesign}
%\renewcommand{\familydefault}{\sfdefault}

\usepackage[letterpaper,left=1.2in,right=1.2in,top=1.2in,bottom=1.2in]{geometry}

% packages i use in essentially every document
\usepackage{graphicx}
\usepackage{enumerate}

% packages i use in many documents but leave off by default
\usepackage{amsmath, amsthm, amssymb}
%\usepackage{amsmath}
% \usepackage{dcolumn}
% \usepackage{endfloat}

% import and customize urls (kjh does this as well, it seems)
\usepackage[usenames,dvipsnames]{color}
\usepackage[breaklinks]{hyperref}

\hypersetup{colorlinks=true, linkcolor=Blue, citecolor=Black, filecolor=Blue,
    urlcolor=Blue, unicode=true}

% add bibliographic stuff 
\usepackage[round]{natbib}
\def\citepos#1{\citeauthor{#1}'s (\citeyear{#1})}
\def\citespos#1{\citeauthor{#1}' (\citeyear{#1})}

% import vc stuff after running `make vc`: \input{vc} \pagestyle{kjhgit}

\begin{document}

\setlength{\parskip}{4.5pt}

\baselineskip 18.5pt

%%\title{The Effects of Copyright Law on the Reuse of Digital Content}

%\title{When Does Copyright Matter? AnThe Impact of Copyright Law on the Reuse of Digital Information}
%\title{Analyzing the differential impact of copyright on digital reuse : An empirical framework}
%\title{An empirical framework for analyzing the reuse of copyrighted material : Evidence from Wikipedia}
%\title{An Empirical Framework for Analyzing the Differential Impact of Copyright on Reuse : }
%\title{An empirical framework for analyzing the differential impact of Copyright}
%\title{An empirical framework for analyzing the differential impact of Copyright : The case of Baseball Digest}

%\title{An Empirical Framework for Analyzing the Impact of Copyright on Creativity}
\title{Analyzing the Impact of Copyright on Creativity -- An Empirical Framework}
%\title{Copyright and Creativty : An Empirical Framework}

\author{Abhishek Nagaraj \footnote{MIT Sloan School of Management; nagaraj@mit.edu}
%        MIT Sloan\\
%        \href{mailto:nagaraj@mit.edu}{\texttt{nagaraj@mit.edu}}\\
%        \\\color{red}EARLY DRAFT \footnote{This document is a preliminary draft and not ready for circulation, significant errors may persist.
}
%}

\date{\today}

\maketitle

\begin{abstract}
What is the impact of copyright on subsequent creativity and what is the role of digitization? Even though copyright governs distribution of creative content in economically important industries like publishing, software, art, movies and music its impact has so far evaded causal empirical analysis. This paper provides a framework that makes such analysis possible focusing on the causal impact of digitization and copyright status on the reuse of underlying information.
Using the digitization of a baseball magazine by Google Books as a case study, I find that post-digitization, copyright negatively impacts the reuse of magazine content on Wikipedia. I further show that copyright's impact on reuse is particularly salient when (a) the underlying content in unique, (b) when it is costly to access and (c) when benefits from its reuse are high. This framework paves the way for estimating the causal impact of copyright in a variety of different domains.
\end{abstract}

% like news, educational and software publishing, photography, genetic and research databases, maps and navigation and historical archives. 

% when do people not use the underlying material?
% - when an alternative is available that does not dimish final value by that much
% - when it is very expensive to access

% access material:
% V(M) - C(a)

% access alternative:
% V(M') - C(a')

%  Specifically, baseball players with images in copyrighted issues of the magazine have a lower probability of having an image on their Wikipedia page as compared to baseball players with images in out-of-copyright issues, after digitization. 
\newpage

\section{Introduction}

\begin{quote}
Copyright is out of control. How -- even if it’s out of control how does it stifle invention? Anybody can make a movie, and the fact that that movie has a copyright, how does that hurt the Internet, for God’s sake?
\begin{flushright}
\emph{Jack Valenti}\\
\emph{Motion Picture Association of America (MPAA)}
\end{flushright}
\end{quote}

The nature of Copyright law is a point of contentious debate. Jack Valenti would like creative works to be protected for ``forever minus one'' days. Proponents of copyright reform, like Harvard Law professor Larry Lessig would claim that the rise of the internet forces one to rethink (and reduce) the power of copyright \citep{lessig_free_2005}. What impact copyright has on creativity and the corresponding role of digital technologies, is a question that matters crucially to the ``core copyright industries'' that include newspapers, books and periodicals, motion pictures, recorded music and music publishing, radio and television broadcasting and software. In the United States alone, the GDP share of these industries was estimated to be at 6.56\%, employing over 5.38 million employees in 2005 \citep{siwek_copyright_2006}. 

Despite the importance of this sector, there is little empirical analysis of the impact of copyright on downstream creativity. Analysis of reuse in the world of copyright is faced with significant problems of data and measurement. Unlike studies of scientific research where citations allows the researcher to construct an ``ideas trail'', the diffusion of downstream media is more informal and harder to track. Further, copyright does not vary much across content unless one is willing to compare considerably older works from the early 20th century and before to more modern works. Even when data for such a comparison can found, causality remains in question because such works are likely to differ in ways that are unobserved to the econometrician and correlated with outcomes of interest. Older books on which data can be found, having survived the test of time, are likely to be of better ``quality'' for example. 

An exception in this area is the work of Paul Heald in the legal literature on the topic (see \cite{heald_property_2007} and \cite{heald_testing_2008}). \cite{heald_testing_2008} attempts a comparison between the top copyrighted and out-of-copyright musical compositions from 1913-32 and looks for instances of use in more recent movies. He finds that out-of-copyright works are equally likely to be used in films as compared to those works that are still under copyright. This research is a promising start even though it suffers from some of the issues highlighted before. \footnote{For example, the fact that most copyrighted works in the sample tend to belong to the golden age of the ``Tin Pan Alley'' between 1926-31 and these songs comprise around 60\% of the works under consideration is problematic.}.

In this paper I present an empirical framework that can be used to estimate the impact of copyright on creativity. 





Using the case study of the digitization of Baseball Digest, a popular baseball magazine, I show how one might use such a framework to estimate the impact of copyright on reuse. Further, this framework is also useful to generate insights about not just if copyright matters, but when it might matter \emph{more} and when it might matter \emph{less}. Finally, an added advantage of the framework in this paper is that it allows the analyst to answer questions about the pertinence of copyright in industries that are facing rapid digitization. Movies are now mostly accessible via streaming services instead of DVDs, books are increasingly digital and the revolution in digital music has already come and gone. The analysis in this paper could potentially be applied to all of these settings.

In the next section I outline the setting for this paper and describe the empirical framework. I then use this framework to estimate the impact of copyright on reuse. I then conclude with an extended discussion of other potential applications of this framework.

\section{Setting}

On December 9 2008, Google Books announced that it would start scanning magazines and would make the digital copies available to readers online\footnote{\href{http://googleblog.blogspot.com/2008/12/search-and-find-magazines-on-google.html}{http://googleblog.blogspot.com/2008/12/search-and-find-magazines-on-google.html}}. One such magazine that was included as a part of this initiative was ``Baseball Digest'', a prominent source of information on the game of baseball. Accordingly since December 2008, over 7 decades of ``Baseball Digest'' have been available for free to readers all over the world.

\begin{quote}
\centering
FIGURE \ref{fig:googlebooks} ABOUT HERE
\end{quote}
 



% In this paper I present

% In the case of books, older out-of-copyright material that is still in print tends to be of higher quality (typically ``classics'' such as Shakespeare or David Copperfield) and what an adequate control group might be for such a sample is an open question. 

% An ideal study would be able to assign copyright status to content that is uncorrelated with its underlying quality or other characteristics like ease-of-access. 

% -- things to worry about
% > over exploitation vs under exploitation
% > incentives to create

% - point out problems
% > very old content
% > causal impact
% > data and measurement


\newpage

\section{Data and Institutional Setting}

\subsection{Google Books and Baseball Digest}

\subsection{Baseball Digest, Wikipedia and Copyright}

\subsection{Suitability of research setting and research design}

\subsection{Data Description}


\section{Analysis}
\subsection{Copyright and the Reuse of Images}
\subsection{Copyright and the Reuse of Text}


\section{Effects on viewership and user contribution}
\subsection{Impact of Baseball Digest on User Contributions}
\subsection{Traffic}

\section{Discussion}


\newpage
% bibliography here

%\renewcommand{\bibsection}{\section{\bibname}\prebibhook}
\baselineskip 14.2pt
\bibliography{/home/nagaraj/copyright/literature/baseball}
\bibliographystyle{apalike}

\newpage
\section{Figures and Tables}

\begin{figure}[h]
\centering
%\includegraphics[scale = 0.35]{/home/nagaraj/baseball/images/bd_jul55cover.png}
\includegraphics[scale = 0.5]{/home/nagaraj/baseball/images/bd_jul2006.png}
\caption{A magazine spread from a digitized Baseball Digest, Issue July 2006}
\label{fig:googlebooks}
\end{figure}

\begin{figure}[h]
\centering
\includegraphics[scale = 0.8]{/home/nagaraj/baseball/images/schematic.png}
\caption{Schematic of Research Design}
\label{fig:schematic}
\end{figure}


\begin{figure}[h]
\centering
%\includegraphics[scale = 0.35]{/home/nagaraj/baseball/images/bd_jul55cover.png}
\includegraphics[scale = 0.5]{/home/nagaraj/baseball/images/cover.png}
\caption{An illustration of how copyright affects knowledge reuse}
\end{figure}

\begin{figure}[h]
\centering
\includegraphics[scale = 0.35]{/home/nagaraj/baseball/images/bd_copyrightcaption.png}
\caption{A screenshot from Wikipedia explaining copyright law pertaining to reuse of material from Baseball Digest}
\end{figure}


\end{document}

